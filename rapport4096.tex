\documentclass[12pt,a4paper]{article}
\usepackage[left=1.5cm,right=1.5cm,top=1.5cm,bottom=1.5cm]{geometry}
\usepackage[utf8]{inputenc}
\usepackage[francais]{babel}
\usepackage[T1]{fontenc}
\usepackage{amsmath}
\usepackage{amsfonts}
\usepackage{amssymb}
\usepackage{textcomp}
\usepackage{listings}
\usepackage{tcolorbox}
\tcbuselibrary{listings,skins}

\author{R. Craciun \and P. Ledoyen}
\title{Projet "Jeu 4096"}
\date{septembre 2016}

\lstdefinestyle{styleC}{
	language=C,
	basicstyle=\ttfamily,
	columns=flexible,
	upquote=true,
	keywordstyle=\bfseries\color{green!40!black},
	identifierstyle=\color{blue},
	stringstyle=\color{orange},
	%numbers=left,
	%numberstyle=\small,
}

\newtcblisting{codeC}[2][]{
    arc=6pt, outer arc=6pt,
    listing only,
    listing style=styleC,
    title=#2,
    #1
    }


\begin{document}
\maketitle

\section{Objectif du projet}
\begin{itemize}
\item Réaliser un jeu de type "2048"
\item Fonder l'organisation du programme sur le modèle CVM
\item Intégrer différentes options et modularités (difficulté, interface, mode de fusion...)
\end{itemize}

\section{Organisation}
Nous avons progressé à deux, sans séparer en deux parties distinctes (ex: d'un côté les calculs, de l'autre l'affichage). Cela nous a permis de mettre en commun rapidement des morceaux de programme,
et de surmonter les problèmes plus efficacement.\\

Nous avons commencé par concevoir l'affichage du plateau de jeu, puis la fusion classique, et à partir de cette base fonctionnelle, nous avons développé les modules supplémentaires.

\section{}

\section{Strucutures de données}
\begin{tabular}{ccc}
\begin{codeC}[hbox,enhanced,drop shadow]{Points}
#include <stdio.h>

int a;
printf("helloworld a=%d", a);

\end{codeC} 

&

\begin{codeC}[hbox,enhanced,drop shadow]{Points}
#include <stdio.h>

int a;
printf("helloworld a=%d", a);

\end{codeC} \\ 

\end{tabular} 


\section{Fonctions importantes}
\subsection{Apparition d'une nouvelle brique}
\subsection{Calcul du meilleur prochain coup}

\section{Conclusion}
\end{document}
